\documentclass{article}
\usepackage[utf8]{inputenc}
\usepackage{graphicx}
\usepackage{float}
\usepackage{optidef}
\usepackage{amsmath}
\usepackage{caption}
\usepackage{subcaption}
\usepackage{mwe}

\title{Quantitative Macroeconomics Homework 3}
\author{Szymon Wieczorek\thanks{In collaboration with Zuzanna Brzóska-Klimek, Sebastian A. Roy and Jakub Blawat}}
\date{October 2019}

\begin{document}
\maketitle
\titlepage

\section*{Exercise 1}

Infinite number of household maximize their utility function:
\begin{maxi}
    {c_t}{E_0{\sum_{t=0}^{\infty}}{\beta}^{t}\ln c_{t}}
    {}{}
    \addConstraint{c_{t} + i_{t} = y_{t}}
    \addConstraint{y_{t} = k_{t}^{1-\theta} (zh_{t})^\theta}
    \addConstraint{i_{t} = k_{t+1} - (1-\delta) k_{t}}
\end{maxi}
We also know that ${\theta} = .067$ and $h_{t} = 0.31$ for all $t$. Additionally, in point a) we have to assume that annual capital-to-output ratio is 4 and investment-to-output ratio is .25. Our role is to calculate transition paths under productivity factor shocks $z$.

\subsection*{a)}

Steady state is computed by the VFI method. In the standard neoclassical growth model (such as the one discussed in this exercise) the only state variable is capital $k_{t}$. Setting capital grid $\{0, 0.05, 0.10, ..., 50\}$ and starting from $k_{init}=25$, after 81 iterations we obtain following results, shown in Table \ref{tab1}:

\begin{table}[h]
    \centering
    \begin{tabular}{c|c|c|c}
         $k_{ss}$ & $y_{ss}$ & $i_{ss}$ & $c_{ss}$  \\
         \hline
         4 & 1 & 0.25 & 0.75 
    \end{tabular}
    \caption{Steady states in the neoclassical growth model with $z=1.63$}
    \label{tab1}
\end{table}

Indeed, we can see that the assumed boundaries for capital are slack, which means that our computations are valid. Moreover, time path of capital is presented in Figure \ref{fig1}.

\begin{figure}
    \centering
    \includegraphics[width=0.6\textwidth]{1.png}
    \caption{Neoclassical growth model capital path}
    \label{fig1}
\end{figure}

\subsection*{b)}

\begin{table}
    \centering
    \begin{tabular}{c|c|c|c}
         $k_{ss}$ & $y_{ss}$ & $i_{ss}$ & $c_{ss}$  \\
         \hline
         8 & 2 & 0.5 & 1.5 
    \end{tabular}
    \caption{Steady states in the neoclassical growth model with $z=3.26$}
    \label{tab2}
\end{table}

As presented in Table \ref{tab2}, choosing different productivity factor $z$ changes nominal values of the steady state, but in equal proportions for all variables (they are simply upscaled). Figure \ref{fig2} shows the updated capital path.

\begin{figure}[p]
    \centering
    \includegraphics[width=0.6\textwidth]{2.png}
    \caption{Neoclassical growth model capital path}
    \label{fig2}
\end{figure}

\subsection*{c)}

Setting $k_{init}=4$, that is the low-productivity steady state value of capital in the high-productivity environment produces convergence paths shown in Figure \ref{tab2}. Since labour supply is fixed, path of labour is omitted (it is flat).

\begin{figure*}
        \centering
        \begin{subfigure}[b]{0.475\textwidth}
            \centering
            \includegraphics[width=\textwidth]{3.png}
            \caption[]%
            {{\small Capital}}    
        \end{subfigure}
        \hfill
        \begin{subfigure}[b]{0.475\textwidth}  
            \centering 
            \includegraphics[width=\textwidth]{4.png}
            \caption[]%
            {{\small Consumption}}    
        \end{subfigure}
        \vskip\baselineskip
        \begin{subfigure}[b]{0.475\textwidth}   
            \centering 
            \includegraphics[width=\textwidth]{5.png}
            \caption[]%
            {{\small Output}}    
        \end{subfigure}
        \quad
        \begin{subfigure}[b]{0.475\textwidth}   
            \centering 
            \includegraphics[width=\textwidth]{6.png}
            \caption[]%
            {{\small Savings}}    
        \end{subfigure}
        \caption[]
        {\small Transition paths from $z=1.63$ to $z=3.26$} 
        \label{fig3}
\end{figure*}
    
\subsection*{d)}

Transition paths presented in Figure \ref{fig4} are calculated under assumption that the initial capital level is $k_{init}=25$. For the first 10 periods agents are allowed to optimize under productivity factor $z=3.26$, but from 11 period on they start using decision rule $g(k_{t})$\footnote{Vector $g(k_{t})$ stores optimal choice of $k_{t+1}$ given $k_{t}$. More technically, it consists of indices of the $\arg \max$ of the $i^{th}$ verse of $\chi$ matrix.} calculated under $z=1.63$.

Interestingly, we can observe that control variables (called also \textit{jumpers}) indeed do not need to be continuous and as a result of the shock they abruptly change their values. In the short run negative productivity shock increases consumption and decreases savings. It affects also slope of the output convergence curve.

\begin{figure}[H]
\centering
\begin{subfigure}{\textwidth}
  \centering
  \includegraphics[width=0.6\linewidth]{7.png}
    \caption{Consumption}
\end{subfigure}

\begin{subfigure}{\textwidth}
  \centering
  \includegraphics[width=0.6\linewidth]{8.png}
  \caption{Output}
\end{subfigure}

\begin{subfigure}{\textwidth}
  \centering
  \includegraphics[width=0.6\linewidth]{9.png}
  \caption{Savings}
\end{subfigure}

\caption{Transition paths under negative productivity shock at $t=11$}
\label{fig4}
\end{figure}

\section*{Exercise 1 - bonus tasks}

\subsection*{e)}

Figures \ref{fig5}, \ref{fig6}, and \ref{fig7} show optimal paths for capital, consumption, output, and savings in an economy with, respectively, consumption tax, capital tax, and both.

Tax rate in both cases has ben set to 20\%. Initial point for capital is 10.

\begin{figure*}[p][H]
        \centering
        \begin{subfigure}[b]{0.475\textwidth}
            \centering
            \includegraphics[width=\textwidth]{10.png}
            \caption[]%
            {{\small Capital}}    
        \end{subfigure}
        \hfill
        \begin{subfigure}[b]{0.475\textwidth}  
            \centering 
            \includegraphics[width=\textwidth]{11.png}
            \caption[]%
            {{\small Consumption}}    
        \end{subfigure}
        \vskip\baselineskip
        \begin{subfigure}[b]{0.475\textwidth}   
            \centering 
            \includegraphics[width=\textwidth]{12.png}
            \caption[]%
            {{\small Output}}    
        \end{subfigure}
        \quad
        \begin{subfigure}[b]{0.475\textwidth}   
            \centering 
            \includegraphics[width=\textwidth]{13.png}
            \caption[]%
            {{\small Savings}}    
        \end{subfigure}
        \caption[]
        {\small Transition paths in economy with consumption tax $\tau_{c}=20\%$} 
        \label{fig5}
\end{figure*}

\begin{figure*}[p][H]
        \centering
        \begin{subfigure}[b]{0.475\textwidth}
            \centering
            \includegraphics[width=\textwidth]{14.png}
            \caption[]%
            {{\small Capital}}    
        \end{subfigure}
        \hfill
        \begin{subfigure}[b]{0.475\textwidth}  
            \centering 
            \includegraphics[width=\textwidth]{15.png}
            \caption[]%
            {{\small Consumption}}    
        \end{subfigure}
        \vskip\baselineskip
        \begin{subfigure}[b]{0.475\textwidth}   
            \centering 
            \includegraphics[width=\textwidth]{16.png}
            \caption[]%
            {{\small Output}}    
        \end{subfigure}
        \quad
        \begin{subfigure}[b]{0.475\textwidth}   
            \centering 
            \includegraphics[width=\textwidth]{17.png}
            \caption[]%
            {{\small Savings}}    
        \end{subfigure}
        \caption[]
        {\small Transition paths in economy with capital tax $\tau_{k}=20\%$} 
        \label{fig6}
\end{figure*}

\begin{figure*}[p][H]
        \centering
        \begin{subfigure}[b]{0.475\textwidth}
            \centering
            \includegraphics[width=\textwidth]{18.png}
            \caption[]%
            {{\small Capital}}    
        \end{subfigure}
        \hfill
        \begin{subfigure}[b]{0.475\textwidth}  
            \centering 
            \includegraphics[width=\textwidth]{19.png}
            \caption[]%
            {{\small Consumption}}    
        \end{subfigure}
        \vskip\baselineskip
        \begin{subfigure}[b]{0.475\textwidth}   
            \centering 
            \includegraphics[width=\textwidth]{20.png}
            \caption[]%
            {{\small Output}}    
        \end{subfigure}
        \quad
        \begin{subfigure}[b]{0.475\textwidth}   
            \centering 
            \includegraphics[width=\textwidth]{21.png}
            \caption[]%
            {{\small Savings}}    
        \end{subfigure}
        \caption[]
        {\small Transition paths in economy with both consumption and capital tax $\tau_{c}=\tau_{k}=20\%$} 
        \label{fig7}
\end{figure*}

\section*{Exercise 2}
\subsection*{a)}
\textbf{Household's maximization problem:}
\begin{maxi}
	{c_l,k_l,h_l}{\frac{c_l^{1-\sigma}}{1-\sigma}-\kappa \frac{h_l^{1+1/v}}{1+1/v}}
	{}{}
	\addConstraint{c = \lambda(w_l(h_l)\eta_l)^{1-\phi_l}+r_l k_l^\eta}+r_{-l} (\tilde{k}_l-k_l)
\end{maxi}
In closed economy foreign investments are equal to zero: $\tilde{k}_l-k_l = 0$. \\
FOCs:
\begin{equation}
    \frac{\partial{(...)}}{\partial{c}} : \frac{1}{c_l^{\sigma}} = \lambda_L \\
\end{equation}
\begin{equation}
    \frac{\partial{(...)}}{\partial{h}} : \kappa h_l^{1/v} = \lambda_L \lambda (w \eta )^{1-\phi_l}(1- \phi_l)h^{- \phi_l}
\end{equation}
Finally we receive Euler equation:
\begin{equation}
    \kappa h^{1/v} = \frac{\lambda(1-\phi)(w\eta)^{1-\phi}}{h^{\eta}}
\end{equation}
We assume that in our economy we have two types of people : low and high productive. So we end up with 2 Euler equations,each for one type. \\
\textbf{Firms problem:}
\begin{maxi}
	{K_l^d,H_l^d}{Z(K_l^d)^{1-\theta}(H_l^d)^{\theta} - w_l H_l^d - r_l K_l^d}
	{}{}
\end{maxi}
FOCs:
\begin{equation}
r = (1-\theta)Z(K^d)^{-\theta}(H^d)^{\theta}    
\end{equation}
\begin{equation}
w = \theta Z K^{1-\theta} H^{\theta - 1}    
\end{equation}

Together with market clearing conditions:
\begin{itemize}
    \item $K = k_h + k_l$
    \item $H = \eta_h h_h +  \eta_lh_l$
\end{itemize}
\textbf{Solving: (in progres)}
Solved using Python.



\end{document}
